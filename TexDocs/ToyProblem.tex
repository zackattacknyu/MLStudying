%%
%% Class homework & solution template for latex
%% Alex Ihler
%%
\documentclass[twoside,11pt]{article}
\usepackage{amsmath,amsfonts,amssymb,amsthm}
\usepackage{graphicx,color}
\usepackage{verbatim,url}
\usepackage{listings}
\usepackage{bbm}
\usepackage{subcaption} 
\usepackage{upquote}
\usepackage[T1]{fontenc}
%\usepackage{lmodern}
\usepackage[scaled]{beramono}
%\usepackage{textcomp}

% Directories for other source files and images
\newcommand{\bibtexdir}{../bib}
\newcommand{\figdir}{fig}

\newcommand{\E}{\mathrm{E}}
\newcommand{\Var}{\mathrm{Var}}
\newcommand{\N}{\mathcal{N}}
\newcommand{\matlab}{{\sc Matlab}\ }

\setlength{\textheight}{9in} \setlength{\textwidth}{6.5in}
\setlength{\oddsidemargin}{-.25in}  % Centers text.
\setlength{\evensidemargin}{-.25in} %
\setlength{\topmargin}{0in} %
\setlength{\headheight}{0in} %
\setlength{\headsep}{0in} %

\renewcommand{\labelenumi}{(\alph{enumi})}
\renewcommand{\labelenumii}{(\arabic{enumii})}

\theoremstyle{definition}
\newtheorem{MatEx}{M{\scriptsize{ATLAB}} Usage Example}

\definecolor{comments}{rgb}{0,.5,0}
\definecolor{backgnd}{rgb}{.95,.95,.95}
\definecolor{string}{rgb}{.2,.2,.2}
\lstset{language=Matlab}
\lstset{basicstyle=\small\ttfamily,
        mathescape=true,
        emptylines=1, showlines=true,
        backgroundcolor=\color{backgnd},
        commentstyle=\color{comments}\ttfamily, %\rmfamily,
        stringstyle=\color{string}\ttfamily,
        keywordstyle=\ttfamily, %\normalfont,
        showstringspaces=false}
\newcommand{\matp}{\mathbf{\gg}}




\begin{document}

\section{Setup}

Given $k$ balls to kick, can either take safe or risky route. \\
Let $f(k)=10k$ be return for safe route\\
\\
Risky route has probability $p$ of success, initial reward $r$ with an increasing reward by $d$\\
\\
Risky strategy will be as follows:\\
Keep pressing it until a success\\
\\
Want to know expected return from this strategy\\

\section{Solving}

Let $N$ be number of tries before successful goal\\
Then 
\[
E[R] = \sum_{n=1}^{\infty} p(N=n) \cdot (r+(n-1)d)
\]
\[
E[R] = \frac{1}{Z} \sum_{n=1}^{\infty} (1-p)^{n-1} \cdot p \cdot (r+(n-1)d)
\]
Z=1 in this case, so we can rewrite as
\[
E[R] = \sum_{n=0}^{\infty} (1-p)^{n} \cdot p \cdot (r+nd)
\]
This simplifies via geometric series to the following
\[
E[R] = r + p \cdot d \cdot \sum_{n=0}^{\infty} n (1-p)^n
\]
Now in general it holds that
\[
\sum_{k=1}^n k z^k = z \frac{1 - (n+1)z^n + n z^{n+1}}{(1-z)^2}
\]
Applying this formula we end up with
\[
E[R] = r + p \cdot d \cdot \frac{1-p}{p^2}
\]
Which simplifies down to
\[
E[R] = r + d \frac{1-p}{p}
\]
For $r=40$, $p=0.2$, and $d=5$, this means $E[R] = 60$\\
For $r=50$, $p=0.1$, and $d=10$, this means $E[R] = 140$\\
For $r=100$, $p=0.05$, and $d=15$, this means $E[R] = 385$


\end{document}
